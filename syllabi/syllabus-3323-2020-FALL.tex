\documentclass[11pt, epsfig]{article}
\usepackage{hyperref}
\usepackage[top=1in, bottom=1in, left=1in, right=1in]{geometry}

%% Future years, change hiere:
\newcommand{\courseterm}[0]{FALL 2020}

\begin{document}

\begin{center}
{\Large\bf CS 3323 PROGRAMMING LANGUAGES\\
  \courseterm~ SYLLABUS}
\end{center}

\paragraph{Instructor:} Martin Kong (DEH-230,  mkong@ou.edu)

\paragraph{Office Hours:} Mondays and Wednesdays 11am - 12:00pm.

% \paragraph{Teaching Assistant and Office Hours:} %Asif M Adnan

\paragraph{Class Meetings:} Adams Hall, 0255, 09:45 AM - 10:35 AM (MWF)

\paragraph{Course Website:} Access through \texttt{https://canvas.ou.edu} 

\paragraph{Delivery Method:}
The course will be delivered by standard face-to-face lectures 
also also live-streamed via Zoom. Lectures will be recorded and uploaded to
the course's cloud, and accessible via a playlist. The links and
access code to the zoom lectures can be found in the Canvas space of the course.
Slides used in the lectures will be available for download in Canvas.

\paragraph{Attendance Policy:}
Class attendance is not mandatory. However, it will probably be more convenient
to be in class for the online quizzes.

\paragraph{Online Access:}

\begin{itemize}

\item 
All instructor office hours will be given via zoom. You can find the access information in the Canvas space of the course.

\item
All TA office hours will be delivered via zoom. Hours TBD. Information will be available in the Canvas space of the course.
\end{itemize}

%\paragraph{Prerequisite:} %CS 3313 Introduction to Compilers;
%Knowledge of Java (or C++). 

\paragraph{Textbook} 
No required book. However, different topics will be covered from the following books:

\begin{itemize}
\item
{\em Programming Languages Pragmatics}, by Michael L. Scott, 4th edition, Morgan Kaufmann.
\item
{\em Compilers: Principles, Techniques, and Tools}, by
Alfred Aho, Ravi Sethi, Jeffrey Ullman, Monica S. Lam. Pearson Publisher.
\item
{\em Engineering a Compiler}, by Keith D. Cooper and Linda Torczon, Morgan Kaufmann.
\end{itemize}


\paragraph{Required Work:}

The final grade consists of:

\begin{itemize}

\item
Quizzes (30\%): Six (6) online quizzes with Canvas. The quiz
with the lowest score will not count towards your final grade.
No make-up up quizzes can be provided. In class attendance
to take the quizzes is not necessary, but might make solving
technical difficulties easier.

In this course, all quizzes are open-book and open-notes, unless otherwise specified.  

\item
Programming Assignments (30\%): You will develop parts of a language
and compiler in four separate programming project submitted via Canvas.
You will have 3-4 weeks for each programming assignment. Can work in groups
of up to 3 students. Working in groups is not mandatory.
See Canvas for specific dates and details.

Late policy: You can submit your assignments up to 3 days late, but each 
group member will lose 1 point per day. No assignments will be accepted past the third day.

\item
Midterm (20\%).

\item
Final Examination (20\%): All course topics included, not just post-midterm.

NOTE1: midterm and final exams will be graded via Gradescope. You are strongly
encouraged to have a laptop both for midterm and final.

NOTE2: midterm and final exams can include topics from the programming assignments.

NOTE3: midterm and final exams will be held in the specified day and time assigned
by the university, unless some event changes this.

\end{itemize}


%The quiz with the lowest scores will not count towards your final grade. 
%No make-up quiz can  be provided.
%We might post quiz solutions on the course website, which usually
%are scans of  student work with name and ID\# removed.
%If you are against posting of your work, please let me know.
% Please submit your homeworks  at \url{https://canvas.ou.edu}.
% One homework of your choice  can be turned in after its due time,
% for which you can earn up to 90\% of credit.
% No other late homework will be accepted.




\paragraph{Learning Goals and Plan:}  
An introduction to theoretical foundations and paradigms of programming
languages. Topics include basic concepts such as lexical analysis, syntax
analysis, syntax-directed translation, type systems and semantics, intermediate code generation, some
practical issues such as  memory management, and  programming paradigms such as
imperative programming, object-oriented programming, functional programming and
scripting.  
%We will cover Chapters 1,2,3,4 and 6 in the first half of the
%semester, and Chapters 7-10, 11 and 14 in the second half. 
%Knowledge of Java (or C++) is required.

\paragraph{ABET Specific Outcomes of Instruction:}
By the end of the semester, the students will be able to
apply computer science theory and software development fundamentals to produce computing-based solutions. 
% 1. Ability to apply knowledge of computing and mathematics appropriate to the discipline (outcome a),
% 2. Ability to use current techniques, skills, and tools necessary for computing practice (outcome i).
For more information, see \url{http://www.abet.org}.

\input{oupolicy.tex}
\end{document}




