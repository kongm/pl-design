\documentclass[11pt, epsfig]{article}
\usepackage{hyperref}
\usepackage[top=1in, bottom=1in, left=1in, right=1in]{geometry}

\begin{document}

\begin{center}
{\Large\bf CS 3323 PROGRAMMING LANGUAGES\\
  SPRING 2020 SYLLABUS}
\end{center}

\paragraph{Instructor:} Martin Kong (DEH-230,  mkong@ou.edu)

\paragraph{Office Hours:} Mondays 11am - 12:30pm; Wednesdays 9am - 10:30am;

% \paragraph{Teaching Assistant and Office Hours:} %Asif M Adnan

\paragraph{Class Meetings:} Physical Science Ctr ( PHSC 0201 ),
03:00 PM - 04:15 PM (MW)

\paragraph{Course Website:} Access through \texttt{https://canvas.ou.edu} 

%\paragraph{Prerequisite:} %CS 3313 Introduction to Compilers;
%Knowledge of Java (or C++). 

\paragraph{Required Text:} Programming Languages Pragmatics, 
Michael L. Scott, 4th edition, Morgan Kaufmann.


\paragraph{Required Work:}
Students are expected to study the chapter before a lecture.
Please submit your homework at {\tt https://canvas.ou.edu}.
Late homework will not be accepted without advance approval.
The final grade consists of

Quiz \hfill 30\%

Homeworks \hfill 20\%

Midterm\hfill 20\% 

Final Examination\hfill 30\% 

Your two quizzes with the lowest scores will not count towards your final grade. 
No make-up quiz can  be provided.
In this course, all quizzes are open-book,
open-notes, unless otherwise specified.  
We will post quiz solutions on the course website, which usually
are scans of  student work with name and ID\# removed.
If you are against posting of your work, please let me know.
% Please submit your homeworks  at \url{https://canvas.ou.edu}.
% One homework of your choice  can be turned in after its due time,
% for which you can earn up to 90\% of credit.
% No other late homework will be accepted.




\paragraph{Learning Goals and Plan:}  
An introduction to theoretical foundations and paradigms
of programming languages. Topics include basic concepts such as 
lexical analysis, syntax analysis, 
type systems and semantics,
some practical
issues such as  memory management and exception
handling, and  programming paradigms such as imperative
programming, object-oriented programming, 
functional programming and scripting. 
We will cover Chapters 1,2,3,4 and 6 
in the first half of the semester, and Chapters 7-10, 11 and 14
in the second half. 
%Knowledge of Java (or C++) is required.

\paragraph{ABET Specific Outcomes of Instruction:}
By the end of the semester, the students will be able to
apply computer science theory and software development fundamentals to produce computing-based solutions. 
% 1. Ability to apply knowledge of computing and mathematics appropriate to the discipline (outcome a),
% 2. Ability to use current techniques, skills, and tools necessary for computing practice (outcome i).
For more information, see \url{http://www.abet.org}.

\input{oupolicy.tex}
\end{document}




