\documentclass[letter,10pt]{article}
\usepackage[left=1in,right=1in,top=1in,bottom=1in]{geometry}
%\usepackage{url}
\usepackage{hyperref}

\title{CS 3323 - Principles of Programming Languages\\ Assignment 2}
\date{}


\begin{document}
\vspace{-1in}
\maketitle

The objective of this programming assignment is to build a fully functional parser. Instructions are provided below.

First, download the files 
grammar.y, scanner.yy, inputs.tar.gz (decompresses to directory {\bf inputs}), Makefile and driver.c from the assignment directory space. Then, perform the necessary modifications
to the grammar's file to accept/reject the example programs provided. You should only work on the grammar file.

The assignment is due on March 2nd, 2020, 11:59pm. Files must be uploaded by then. Late policy deduction applies.

\begin{enumerate}

\item (0.5pt)
Complete the production corresponding to the {\bf read} non-terminal. It should produce a comma-separated list of
variable references ({\bf varref}). The list of variable references should be of at least length one.

\item (0.5pt)
Complete the production corresponding to the {\bf expr\_list} non-terminal. It should produce a comma-separated list of
arithmetic expressions ({\bf a\_expr}). The list of arithmetic expressions should be of at least length one.

\item (0.5pt)
Define three productions for the non-terminal {\bf l\_fact}: 
  
  \begin{itemize}
  \item a left-recursive rule producing comparisons of arithmetic expressions ({\bf a\_expr} non-terminal). It should use the {\bf oprel} non-terminal already defined.
  \item a single arithmetic expression.
  \item A logical expression in parenthesis ({\bf l\_expr} non-terminal).
  \end{itemize}

\item (1pt)
Define five productions for the non-terminal {\bf a\_fact} based on the following description:

  \begin{itemize}
  \item An {\bf a\_fact} can be a variable reference (non-terminal {\bf varref}).
  \item The token T\_NUM.
  \item A literal string (token T\_LITERAL\_STR).
  \item The non-terminal {\bf a\_fact} preceded by the T\_SUB token (Note: Do not use '-').
  \item A parenthesized arithmetic expression.
  \end{itemize}

\item (0.5pt)
Define two productions for the {\bf varref} non-terminal that match the below description:

  \begin{itemize}
  \item A variable reference can be the T\_ID token.
  \item A variable reference can be a left-recursive list of arithmetic expressions delimited by '[' and ']'. The recursion terminates with the T\_ID token (See above description).
  \end{itemize}

\item (2pt)
Complete the control-flow constructs. Observe that a statement list surrounded by T\_BEGIN and T\_END is also a statement. The non-terminal {\bf l\_expr} must be
used for representing logical expressions. Use test cases for*.smp, if*.smp, repeat*.smp and for*.smp.

  \begin{itemize}
  \item {\bf foreach}: Complete the partially-defined production. See input cases for[1-4]\_pass.smp.
  \item {\bf repeat-until}: 
        Define it  as a list of statements. Use the non-terminal {\bf stmt\_list}). The list must be  delimited by the tokens T\_REPEAT and T\_UNTIL. 
        The controlling condition should use the {\bf l\_expr} non-terminal. Do not add parentheses.
  \item {\bf while}:
        The T\_WHILE token followed by a logical expression and any statement.
  \item {\bf if-then/if-then-else}:
        The T\_IF token followed by a logical expression (non-terminal {\bf l\_expr}). The true branch should be a statement preceded by the T\_THEN token,
        whereas the T\_ELSE branch can either be empty or start with the T\_ELSE token followed by a statement.
  \end{itemize}

\end{enumerate}

For convenience, a Makefile is provided, but you are not required to use it.

To rebuild the binary ({\bf simple.exe}) run: make all

To test a single input file, run: ./simple.exe $<$ inputfile.smp

Several online resources can be found in the web, for instance:
\begin{itemize}
\item
\url{https://www.gnu.org/software/bison/manual}
\item
\url{https://www.lysator.liu.se/c/ANSI-C-grammar-y.html#multiplicative-expression}
\end{itemize}

More resources can be found by searching for the key terms: yacc/bison parser generator.

Do not change the driver file, nor the scanner.yy files. Do not print anything to the output.

Every student should {\bf upload a single file named: ABCDEFGHI.y}, where ABCDEFGHI is the 9-digit code identifying the student (not the 4+4).

\end{document}
