\documentclass[letter,10pt]{article}
\usepackage[left=1in,right=1in,top=1in,bottom=1in]{geometry}
%\usepackage{url}
\usepackage{hyperref}

\title{CS 3323 - Principles of Programming Languages\\ Assignment 3}
\date{}


\begin{document}
\maketitle

The objective of this programming assignment is to build a functional compiler with basic I/O support and arithmetic computations. Instructions are provided below.

First, download the files 
grammar.y, scanner.yy, icode.cc, icode.hh, symtab.cc, symtab.hh, inputs-int.tar.gz, inputs-float.tar.gz, Makefile, run-all.sh and driver.cc from the assignment directory space. 
Then, perform the necessary modifications
to grammar.y and to icode.cc to add support for floating point operations. You should only work on the grammar file (grammar.y)
and in icode.cc.

The assignment is due on April 1st, 2020, 11:59pm. Both modified files (grammar.y and icode.cc) must be uploaded by then. Late policy deduction applies.

Perform the following modifications to the compiler:

\begin{enumerate}

\item (2.0pt)
In grammar.y, complete the semantic actions corresponding to the floating point arithmetic binary operations 
of the non-terminals {\textit a\_expr} and {\textit a\_term}: (OP\_FADD, OP\_FSUB, OP\_FMUL and OP\_FDIV).
You should query the {\bf datatype} field of a temporary symbol to determine the correct operation to be generated.

\item (1.0pt)
In icode.cc, function {\bf run}, implement the run-time actions for the floating point arithmetic binary operations
OP\_FADD, OP\_FSUB, OP\_FMUL and OP\_FDIV. These four operations are specific to the datatype DTYPE\_FLOAT.

\item (0.5pt)
In icode.cc, function {\bf run}, complete the run-time action of the floating point arithmetic unary operation
OP\_UMIN for the DTYPE\_FLOAT datatype ({\bf float}). The current implementation only supports DTYPE\_INT (a C/C++ int). 

%\item (1.0pt)
%In grammar.y, complete the semantic actions corresponding to the binary arithmetic operations 
%of the non-terminals {\textit a\_expr} and {\textit a\_term} by
%validating that the {\bf datatype} fields of both the left and right binary operands are identical.
%If that is not the case, invoke the C function {\bf exit(1);} (declared in stdlib.h).

\item (0.5pt)
In icode.cc, function {\bf run}, complete the run-time action of the operation OP\_LOAD\_CST for the datatype DTYPE\_FLOAT.
It should be very similar to the DTYPE\_INT case.

\item (0.5pt)
In icode.cc, function {\bf run}, complete the run-time action of the operation OP\_LOAD for the datatype DTYPE\_FLOAT.
It should be very similar to the DTYPE\_INT case.

\item (0.5pt)
In icode.cc, function {\bf run}, complete the run-time action of the operation OP\_STORE for the datatype DTYPE\_FLOAT.
It should be very similar to the DTYPE\_INT case.


\end{enumerate}

A number of test cases are provided, both for the {\bf int} (directory {\bf inputs-int}) and {\bf float} (directory {\bf inputs-float}) 
language datatypes.
The current implementation of the provided compiler works with all the test cases of the directory {\bf inputs-int}.
You can test your progress with the files in {\bf inputs-float}. Note that the output should be practically identical to
the corresponding {\bf int} test case.

For convenience, a Makefile is also provided, but you are not required to use it.
The Makefile will build two binaries, {\bf simple.exe} and {\bf simple-debug.exe}. The latter will output
the symbol table, instruction table and a number of debug prints. Grading will be performed with
{\bf simple.exe} . If you need to add debug printing information, always enclose it between "\#ifdef \_SMP\_DEBUG\_" and "\#endif".

To test a single input file, run: ./simple.exe $<$ inputfile.smp

You can also test {\bf all} the test cases of a single directory with the script {\bf run-all.sh}.
It expects the directory name to test.

Several online resources can be found in the web, for instance:
\begin{itemize}
\item
\url{https://www.gnu.org/software/bison/manual}
\item
\url{https://www.lysator.liu.se/c/ANSI-C-grammar-y.html#multiplicative-expression}
\end{itemize}

More resources can be found by searching for the key terms: yacc/bison parser generator.

Do not change any other file. Do not print anything to the output outside of the conditional compilation directives.

Every student should {\bf upload the two modified files in a directory named: ABCDEFGHI}, where ABCDEFGHI is the 9-digit code identifying the student (not the 4+4).
You should only upload the files grammar.y and icode.cc.

\end{document}
