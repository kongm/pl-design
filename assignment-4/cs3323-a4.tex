\documentclass[letter,10pt]{article}
\usepackage[left=1in,right=1in,top=1in,bottom=1in]{geometry}
%\usepackage{url}
\usepackage{hyperref}

\title{CS 3323 - Principles of Programming Languages\\ Assignment 4}
\date{}


\begin{document}
\maketitle

The objective of this programming assignment is to add basic control-flow functionality 
to a simple compiler being developed in this course. Instructions are provided below.

First, download the files 
grammar.y, scanner.yy, icode.cc, icode.hh, symtab.cc, symtab.hh, inputs-outputs.tar.gz, Makefile, run-all.sh and driver.cc 
from the assignment directory space. 
Then, perform the necessary modifications
to grammar.y to add support for the repeat-until, while-do and if-then-else programming constructs. 
You should only work on the grammar file (grammar.y).

The assignment is due on April 29th, 2020, 11:59pm. Late policy deduction applies.

Perform the following modifications to the compiler:

\begin{enumerate}

\item (0.5pt)
This version of the compiler introduces 5 new intermediate code operations: OP\_LT, OP\_GT,
OP\_JMP, OP\_JNZ and OP\_JZ. See the file icode.cc, lines 296--354.
Read their implementation to be sure you understand their semantics.
The corresponding constant definitions and instruction codes have been added to icode.hh,
and can be used as new intermediate code operations.

Also observe the new rule l\_expr in grammar.y, which allows to compare two integer
expressions and store the result of this comparison in an intermediate variable.
The action stores a pointer to the intermediate variable generated for any other
rule to use it (e.g. repeat-until).

A few small modifications have been performed to the loop executing the intermediate
operations, and they mainly pertain to the execution of conditional and unconditional
jumps: changing the program counter (now explicitly {\bf pc}) by some instruction
position given by an argument of the intermediate code (always the 3rd argument).

\item (1.5pt)
In grammar.y, complete the semantic actions  for the repeat-until construct (see construct\_repeat rule).
The construct {\bf jumps backwards} to the first instruction of the loop's body whenever the condition is false.
This means that you must use the new intermediate code operation OP\_JZ (jump if zero).
As the jump target is to an operation preceding the same OP\_JZ, you will need to store the target
address in the parser's stack. This can be done by assigning the next instruction to be generated
(see macro INSTRUCTION\_NEXT) to the variable {\bf @\$.begin.line}.
To retrieve the value stored in some semantic action, use {\bf @X.begin.line}, where X is the position
of the semantic action as a symbol in the right-hand-side of grammar's rule. For instance,
to retrieve an integer value stored in the stack by a semantic action occupying the second position
as a symbol, use: {\bf @2.begin.line}. This form of accessing the stack essentially replaces the simpler
older form of {\bf \$2}.

\item (1.5pt)
In grammar.y, complete the semantic actions  for the while-do construct (see construct\_while rule).
Observe that the rule of this construct has 3 semantic actions. The first one precedes the condition
evaluation. The second comes after the condition, but before the loop's body. The last semantic action
takes places after recognizing and generating the code for the loop's body.
The while-do construct potentially {\bf jumps forward}, right after evaluating the loop's condition, 
to the first instruction following the loop's body whenever the condition is false.
This means that you must use the new intermediate code operation OP\_JZ (jump if zero).
Since the jump target is to an operation that has not been generated at the moment that OP\_JZ is being created, 
you will need to store the instruction entry associated to the jump, and complete it later in the third 
semantic action. The entry number to completed can be stored in the 
parser's stack. This can be done by assigning the last generated instruction number
(see macro INSTRUCTION\_LAST) to the variable {\bf @\$.begin.line}.

\item (1.5pt)
In grammar.y, complete the semantic actions to implement the if-then and if-then-else constructs.
See rule construct\_if, which will use 3 semantic actions. The first one must generate
a jump to the false-branch of the construct when the condition is false. Also store
the instruction entry in the parser's stack as you will need it to complete the destination of the
jump. The second action performs two tasks: i) generate an unconditional jump to potentially
skip the execution of the else-branch, and ii) complete the jump's destination generated
in the first semantic action. The third semantic action sets the target jump address for the
unconditional jump created in the second semantic action. Be carefully to distinguish
between {\bf the last generated instruction entry} (see macro INSTRUCTION\_LAST) and
{\bf the next instruction to be generated} (see macro INSTRUCTION\_NEXT).


\end{enumerate}

A number of test cases are provided (see inputs-outputs.tar.gz). 
If you write your own test cases, limit them to using only the {\bf int}
data type of our language. Also note that the current version of this compiler does not support $<=$, $>=$, $==$ nor $!=$ (as in C syntax).

For convenience, a Makefile is also provided, but you are not required to use it.
The Makefile will build two binaries, {\bf simple.exe} and {\bf simple-debug.exe}. The latter will output
the symbol table, instruction table and a number of debug prints. Grading will be performed with
{\bf simple.exe} . If you need to add debug printing information, always enclose it between "\#ifdef \_SMP\_DEBUG\_" and "\#endif".

To test a single input file, run: ./simple.exe $<$ inputfile.smp

{\bf Grading will be performed based on the output of your compiler}. Your output should match
exactly the one in the .out files. Each .smp file has a corresponding output file. See the contents
of the inputs-outputs directory (after decompressing the tar.gz file).

You can also test {\bf all} the test cases of a single directory with the script {\bf run-all.sh}.
It expects the directory name to test.

Several online resources can be found in the web, for instance:
\begin{itemize}
\item
\url{https://www.gnu.org/software/bison/manual/bison.html}
\end{itemize}

More resources can be found by searching for the key terms: yacc/bison parser generator.

Do not change any other file. Do not print anything to the output outside of the conditional compilation directives.

{\bf Each student should  upload their modified grammar.y renamed to ABCDEFGHI.y}, where ABCDEFGHI 
is the 9-digit code identifying the student (not the 4+4).

\end{document}
