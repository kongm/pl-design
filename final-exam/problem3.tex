

\paragraph{Problem 3: Writing to an array cell (10 points)}

As next step, we modify the assignment rule of grammar.y from the below form:

\vspace{2em}

\begin{lstlisting}

assignment : varref T_ASSIGN a_expr 
      {
        itab_instruction_add (itab, OP_STORE, $1->addr, $1->datatype, $3->addr);
        $$ = $1;
      }

\end{lstlisting}

\vspace{1em}

to its new form:

\vspace{1em}

\begin{lstlisting}
assignment : varref '[' a_expr ']' T_ASSIGN a_expr 
  {
    /* You will add more code here */ 
  }
\end{lstlisting}

\vspace{2em}

{\bf Problem 3A (3 points)}
\\
Briefly argue why we had to change the assignment rule in our grammar.
(HINT: the reason has to do with certain modifications to the non-terminal \texttt{varref}).

\begin{tcolorbox}[height=4in]

\end{tcolorbox}


\pagebreak
{\bf Problem 3B (4 points)}
\\
Now, briefly describe how would you implement a new 
{\bf \texttt{OP\_STORE\_ARRAY\_CELL}} intermediate code operation 
in the \texttt{run ()} function of icode.cc.
This operation is somewhat similar to \texttt{OP\_STORE} (See lines 144-150 in icode.cc), 
but it must store 
a single value in a particular array cell.
You can inspire yourself from your own answer of Problem 2B. Your
answer to this question must also be consistent with your answer in Problem 2A.

\begin{tcolorbox}[height=6in]

\end{tcolorbox}

\pagebreak

{\bf Problem 3C (3 points)}
\\
Finally, complete the semantic actions of the new assignment rule  below:

\begin{lstlisting}
assignment : varref '[' a_expr ']' T_ASSIGN a_expr 
  {
    /* You will add more code here */ 
  }
\end{lstlisting}

You must use the new intermediate code operation
{\bf \texttt{OP\_STORE\_ARRAY\_CELL}} in the above semantic action. You don't need to worry
about how are now regular assignments performed (e.g. \texttt{a := 10}).
You can write your answer in words or in pseudocode, whatever is easier.

\begin{tcolorbox}[height=6in]

\end{tcolorbox}

