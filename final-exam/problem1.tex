
\paragraph{Problem 1: Declaring 1-dimensional arrays (10 points)}
The following is an excerpt from the grammar.y file (both from the 3rd and 4th assignment),
which permits to declare scalar variables (non-arrays) with data types:

\begin{lstlisting}
declaration: datatype T_ID 
  { 
    assert (symtab);
    assert (itab);
    symbol_t * sym = symbol_create (symtab, $2, $1); 
    assert (sym);
    symbol_add (symtab, sym);
  }
    ;
\end{lstlisting}

First, recall that function \texttt{symbol\_create} adds a new symbol
to the symbol table. The datatype used to create the new symbol is obtained from the {\texttt datatype} non-terminal
in the right-hand side of the rule, while the name of the symbol is the string
recognized together with the \texttt{T\_ID} token.

Suppose now that the declaration of variables is modified by adding a non-terminal \texttt{array\_size} 
(see below). You can observe that it expects a number such as 10 or 5.  If, however, that number
is omitted, then the semantic action is to return 1 via the parser's stack (assignment \$\$ = 1). 
Obviously, the datatype of
the non-terminal \texttt{array\_size} will be \texttt{int}, similar to lines 58, 66--67 in grammar.y.
(NOTE: T\_INTEGER is the token returned by the scanner when an integer number is recognized,
whereas T\_DT\_INT is the token associated to the keyword \texttt{int} in our language.)

\begin{lstlisting}
declaration: datatype array_size T_ID 
  { 

  }
  ;

array_size : T_INTEGER { $$ = $1; }
          |   { $$ = 1; }
          ;
        
\end{lstlisting}


{\bf Problem 1A (2 points):}
\\
Refer to the definition of \texttt{struct simple\_symbol} in symtab.hh.
The current data structure for declaring variables only has 3 fields: name, addr and datatype.
This is not sufficient for declaring 1-dimensional arrays. In 5 or fewer lines, describe why
is the current data structure not adequate for supporting 1-dimensional arrays.


\begin{tcolorbox}[height=3in]

\end{tcolorbox}

\pagebreak
{\bf Problem 1B (2 points):}
\\
Propose one or more modifications
to the {\texttt simple\_symbol} data structure to be able to declare 1-dimensional arrays. 
You can assume that you
only have to support arrays of integers. 

\begin{tcolorbox}[height=3in]

\end{tcolorbox}



{\bf Problem 1C (2 points):}
\\
Describe now the modifications you would have to perform on \texttt{symbol\_create} in order to use
it in the modified grammar. You don't need to write code, but if it helps to explain yourself,
you can re-write part or all of \texttt{symbol\_create} to describe the changes you need to do.

\begin{tcolorbox}[height=4in]

\end{tcolorbox}


\pagebreak
{\bf Problem 1D (2 points):}
\\
Now that you have added support for the new symbol data structure and symbol creation, describe in
words the steps or tasks to perform in the new semantic action. You can ignore the declaration
of scalar (regular) variables.
%Recall that it should allow
%the declaration of scalars and of 1-dimensional arrays

\begin{lstlisting}
declaration: datatype array_size T_ID 
  { 

  }
  ;
\end{lstlisting}

\begin{tcolorbox}[height=4in]

\end{tcolorbox}


{\bf Problem 1E (2 points):}
\\
Given the grammar definition, would it be possible to declare an array in the following way?

\begin{lstlisting}
int n myarray;
\end{lstlisting}


Briefly justify your answer.

\begin{tcolorbox}[height=2in]

\end{tcolorbox}
