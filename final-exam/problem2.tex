


\paragraph{Problem 2: Reading from an array cell (10 points)}


Refer to the grammar.y file of assignment \#3, lines 199--206 and lines 239-244,
which define how a value stored at a variable is retrieved:

\vspace{2em}

\begin{lstlisting}
a_fact : varref 
      {
        symbol_t * res;
        assert ($1 && "Did not find variable");
        res = make_temp (symtab, $1->datatype);
        itab_instruction_add (itab, OP_LOAD, res->addr, $1->datatype, $1->addr);
        $$ = res;
      }
\end{lstlisting}

\vspace{2em}


\begin{lstlisting}
varref : T_ID 
    {
      symbol_t * sym = symbol_find (symtab, $1);
      assert (sym && "Ooops: Did not find variable!");
      $$ = sym;
    }
\end{lstlisting}

\vspace{2em}


To be able to read from 1-dimensional array cells, we modify the non-terminal \texttt{varref}
in the following way:

\vspace{2em}


\begin{lstlisting}
varref : T_ID index
    {
      symbol_t * sym = symbol_find (symtab, $1);
      assert (sym && "Ooops: Did not find variable!");
      /* You will add more code here */
      $$ = sym;
    }
index : '[' a_expr ']' { $$ = $2; }
    |    { $$ = NULL; }
    ;
\end{lstlisting}

\vspace{2em}

The above rules define a non-terminal \texttt{index}, which has two
rules associated to it. The first rule stores the symbol of the intermediate
variable that stores the index being accessed. The second rule is recognized
by the parser when no index expression is found. In that case, the null pointer
is stored in the parser's stack.

You should notice that in the original grammar, the loading of a variable's value
was performed on one of the rules of the non-terminal \texttt{a\_fact} via the
\texttt{OP\_LOAD} intermediate operation. The objective now is to move the
work being performed in \texttt{a\_fact}'s semantic action to the non-terminal \texttt{varref}.


Refer to files icode.hh and icode.cc of programming assignment \#3, in particular, the
\texttt{run ()} function and the \texttt{OP\_LOAD} intermediate operation.
You are asked to extend the intermediate code generation process to support loading
the value of a single array cell into a temporary variable. To do so, we will
add a new intermediate operation named {\bf \texttt{OP\_LOAD\_ARRAY\_CELL}}.
The following parts of this problem will guide you through the steps of reading
values of specific array cells into a temporary variable.

\pagebreak

{\bf Problem 2A (3 points):}
\\
Decide whether you need or not to change the definition of the 
\texttt{simple\_icode} data structure (See icode.hh).
If you decide to modify it, describe how. List the new fields you are adding,
  together with their datatype and describe how you intend to use it/them.
If you decide to not modify the data structure say so, but your answer will have to 
be consistent with the subsequent parts of this problem.

\begin{tcolorbox}[height=3in]

\end{tcolorbox}

\vspace{2em}

{\bf Problem 2B (4 points):}
\\
Briefly describe how will you implement 
the {\bf \texttt{OP\_LOAD\_ARRAY\_CELL}} intermediate code operation 
in the \texttt{run ()} function of icode.cc.
Describe the semantics of each of the fields of the \texttt{simple\_icode} field
for the new operation {\bf \texttt{OP\_LOAD\_ARRAY\_CELL}}.
Mention what a field represents, e.g. a variable, an address in memory, the source of the
load, or the target of the load. 

\begin{tcolorbox}[height=3in]

\end{tcolorbox}

\pagebreak

{\bf Problem 2C (3 points):}
\\
Now that you have implemented the new intermediate operation, complete the semantic
action of the new \texttt{varref} rule (``You will add more code here'') by
calling 
the new operation {\bf \texttt{OP\_LOAD\_ARRAY\_CELL}}
:

\begin{lstlisting}
varref : T_ID index
    {
      symbol_t * sym = symbol_find (symtab, $1);
      assert (sym && "Ooops: Did not find variable!");
      /* You will add more code here */
      $$ = sym;
    }
\end{lstlisting}

Recall that the new rule should behave as follows:
\begin{itemize}
\item If the index actually appears, then some specific entry of the array 
must be read and stored in a temporary variable.
\item If no index is found, then the rule above should behave as a regular 
\texttt{OP\_LOAD} operation (e.g. \texttt{\_T10 = a}).
\end{itemize}

You can write your answer in words or in pseudocode, whatever is easier.

\begin{tcolorbox}[height=5in]

\end{tcolorbox}

