\documentclass{article}

\usepackage{afterpage}
\usepackage{booktabs} % For formal tables
\usepackage{setspace}
\usepackage[multiple]{footmisc}
\usepackage{url}
\usepackage{amsmath,amssymb}
\usepackage{alltt}
%\usepackage{lstlisting}
\usepackage{xspace}
\usepackage{epsfig}
\usepackage{graphics}
\usepackage{bbold}
\usepackage{empheq}
\usepackage{tcolorbox}
\usepackage[left=1in,right=1in,top=1.2in,bottom=1.2in,%
            footskip=.5in]{geometry}
%\usepackage[table]{xcolor}
\usepackage{colortbl,hhline}
%\usepackage{pgf}
%\usepackage{pgfplotstable}
\usepackage{multirow}
%\definecolor{lightgreen}{HTML}{90EE90}

\usepackage{amsmath,amsfonts,amsbsy}
\usepackage{fancyvrb,keyval,ifthen}
\usepackage{graphicx}

\usepackage{booktabs}
\usepackage{listings}
\lstset{basicstyle=\ttfamily\footnotesize, frame=single}

\usepackage{paralist}
\usepackage{enumitem}
\setlist{nolistsep}
\usepackage{titlesec}
\usepackage{balance}
\usepackage{array}

\usepackage[linesnumbered,ruled,vlined]{algorithm2e}
%% %\makeatletter
%% %\renewcommand{\ALG@beginalgorithmic}{\footnotesize}
%% %\makeatother
%% \SetAlFnt{\footnotesize}

%% \usepackage{fancyhdr}


\setlength{\oddsidemargin}{0pt}

\title{CS-3323 Principles of Programming Languages -- Final Exam}

\begin{document}

\date{May 1st -- May 6th, 2020}

\maketitle
{\bf\large
Name: \underline{\mbox{\hspace{4cm}}} 
}
\vspace{2em}

{\bf\large
ID\# \underline{\mbox{\hspace{4cm}}} \hfill
Score: \underline{\mbox{\hspace{4cm}}}
}

\vspace{4em}

\noindent
{\bf Instructions (READ CAREFULLY):}

\begin{itemize}

\item
This exam is worth a total of 30 points. 

\item 
An extra credit question worth 5 points is also included. 

\item You have until May 6th, 6:30pm to upload your solution.

\item You can use books, slides, handouts, and any sort of notes.

\item You can work in groups of up to 3 students, that is, yourself and two more.

\item {\bf All students must upload the same solution by the deadline, both in Gradescope and in Canvas (Final Exam directory at the root of the file space)}.

\item Each student must upload a single PDF file with all the answers. One possible way is printing
a copy of the exam, filling it (either by hand or electronically), and then scanning it. If you don't
have access to a scanner, you can take a picture of each page and then transfer the photos to a computer.
Regardless of the source device (scanner or camera/phone), consolidate all solutions and pages in a single MS WORD file (or similar word editor)
and export it to a PDF format.

\item The estimated time to solve this exam is 4 hours. However, preparing the PDF file with the solutions will likely take time. Please consider the
overhead of scanning/etc in order to submit your exam solutions by the deadline.

\item If you have any question or if something is unclear, email me. 

\item If your question is deemed as a way of luring me to give you a solution to the exam problems, you will be outright ignored.


\end{itemize}

\vspace{2em}

Before proceeding, disclose the students of your group. List their names and student ID (9-digit number):

\begin{tcolorbox}[height=1.5in]

\end{tcolorbox}

\pagebreak

Discussions with peer classmates is encouraged, but your solution can only be
identical to that of the members in your group.  Disclose below the classmates
outside your group (name and ID) with whom you have exchanged opinions,
arguments and/or rationale of solutions for this exam:

\begin{tcolorbox}[height=3in]

\end{tcolorbox}


\vspace{1em}

{\bf Context:}

\begin{itemize}

\item
In programming assignments 3--4 you developed a simple compiler with support for
integer and floating point arithmetic, and basic control-flow constructs.
However, the language and compiler did not support any sort of array data type.

\item
The following exam questions will guide you through the process of adding support
for 1-dimensional integer arrays at both the language level and at the compiler
level.

\item
You can refer to the file grammar.y and all other files of the 3rd and 4th programming assignments for
more precise implementation details. You can find all the files in Canvas.

\item
You are also free to modify the files in order to test that your solution is correct or makes
sense. However, that is not a requirement for getting full credit in the exam.

\end{itemize}


\pagebreak

\vspace{1cm}

\noindent
{\bf Problem 1 (5pts)}
\vspace{1cm}


\noindent
(1pt) Assuming right associativity for the MINUS binary operator (-), left-associativity for exponentiation (**),
and ** having a higher precedence than -,
show the evaluation steps and the final result of evaluating the expression: 2 ** 6 ** 0.5 - 9 - 1

\vspace{1in}

\noindent
(1pt) Consider the regular expression: $(a|b) * c+ (bc | dc)*$ \\
\\
  Write two strings that would be rejected and two string that would be accepted by the above regular expression.
  All strings must be at least 4 symbols long.


\vspace{1in}

\noindent
(0.5pt)  How many unique / distinct tokens would a C scanner produce for the following code snippet

\begin{lstlisting}
while ( a + b < 10 ) b = b + 2 ;
\end{lstlisting}

\vspace{1in}

\pagebreak
\noindent
(1pt) For the below statements, mark either True or False:

\vspace{1cm}

\begin{tabular}{|p{13cm}|p{2cm}|}
\hline
{\bf Statement} & {\bf True/False} \\
\hline
Left/right operator associativity is necessary for the + operator &  \\
\hline
LL parsing works in a bottom-up fashion  &   \\
\hline
Operator associativity deals with the order of evaluation of a sequence of operations of the exact same type	 &  \\
\hline
LL parsers are perfectly suitable to handle left-recursion &  \\
\hline
\end{tabular}

\vspace{1cm}

\noindent
(0.5pt) Write a regular expression that would accept the following 4 strings:

\begin{itemize}
\item
a b
\item
a a b b c c
\item
a b b b c c c c
\item
b b b b b b c c c c c
\end{itemize}

\vspace{1in}


\noindent
(1.0pt) 
Write a small grammar that permits to produce lists of function arguments for a C-like language.  
The tokens to use in this grammar are: ID, INT, FLOAT, LEFTPAR '(', RIGHTPAR ')' and COMMA ' , '. $\epsilon$ is the empty string.

\noindent
You can use the pipe character '$|$' to separate multiple right-hand-sides for the same non-terminal.
The grammar can have either left-recursion or right-recursion, but not both types.

Below are 3 examples, one per line, of the strings that could  be accepted:

\begin{itemize}
\item
( int a, int b )

\item
( int a, float b, int c)

\item
( )
\end{itemize}

\vspace{2in}


\pagebreak



\paragraph{Problem 2: Reading from an array cell (10 points)}


Refer to the grammar.y file of assignment \#3, lines 199--206 and lines 239-244,
which define how a value stored at a variable is retrieved:

\vspace{2em}

\begin{lstlisting}
a_fact : varref 
      {
        symbol_t * res;
        assert ($1 && "Did not find variable");
        res = make_temp (symtab, $1->datatype);
        itab_instruction_add (itab, OP_LOAD, res->addr, $1->datatype, $1->addr);
        $$ = res;
      }
\end{lstlisting}

\vspace{2em}


\begin{lstlisting}
varref : T_ID 
    {
      symbol_t * sym = symbol_find (symtab, $1);
      assert (sym && "Ooops: Did not find variable!");
      $$ = sym;
    }
\end{lstlisting}

\vspace{2em}


To be able to read from 1-dimensional array cells, we modify the non-terminal \texttt{varref}
in the following way:

\vspace{2em}


\begin{lstlisting}
varref : T_ID index
    {
      symbol_t * sym = symbol_find (symtab, $1);
      assert (sym && "Ooops: Did not find variable!");
      /* You will add more code here */
      $$ = sym;
    }
index : '[' a_expr ']' { $$ = $2; }
    |    { $$ = NULL; }
    ;
\end{lstlisting}

\vspace{2em}

The above rules define a non-terminal \texttt{index}, which has two
rules associated to it. The first rule stores the symbol of the intermediate
variable that stores the index being accessed. The second rule is recognized
by the parser when no index expression is found. In that case, the null pointer
is stored in the parser's stack.

You should notice that in the original grammar, the loading of a variable's value
was performed on one of the rules of the non-terminal \texttt{a\_fact} via the
\texttt{OP\_LOAD} intermediate operation. The objective now is to move the
work being performed in \texttt{a\_fact}'s semantic action to the non-terminal \texttt{varref}.


Refer to files icode.hh and icode.cc of programming assignment \#3, in particular, the
\texttt{run ()} function and the \texttt{OP\_LOAD} intermediate operation.
You are asked to extend the intermediate code generation process to support loading
the value of a single array cell into a temporary variable. To do so, we will
add a new intermediate operation named {\bf \texttt{OP\_LOAD\_ARRAY\_CELL}}.
The following parts of this problem will guide you through the steps of reading
values of specific array cells into a temporary variable.

\pagebreak

{\bf Problem 2A (3 points):}
\\
Decide whether you need or not to change the definition of the 
\texttt{simple\_icode} data structure (See icode.hh).
If you decide to modify it, describe how. List the new fields you are adding,
  together with their datatype and describe how you intend to use it/them.
If you decide to not modify the data structure say so, but your answer will have to 
be consistent with the subsequent parts of this problem.

\begin{tcolorbox}[height=3in]

\end{tcolorbox}

\vspace{2em}

{\bf Problem 2B (4 points):}
\\
Briefly describe how will you implement 
the {\bf \texttt{OP\_LOAD\_ARRAY\_CELL}} intermediate code operation 
in the \texttt{run ()} function of icode.cc.
Describe the semantics of each of the fields of the \texttt{simple\_icode} field
for the new operation {\bf \texttt{OP\_LOAD\_ARRAY\_CELL}}.
Mention what a field represents, e.g. a variable, an address in memory, the source of the
load, or the target of the load. 

\begin{tcolorbox}[height=3in]

\end{tcolorbox}

\pagebreak

{\bf Problem 2C (3 points):}
\\
Now that you have implemented the new intermediate operation, complete the semantic
action of the new \texttt{varref} rule (``You will add more code here'') by
calling 
the new operation {\bf \texttt{OP\_LOAD\_ARRAY\_CELL}}
:

\begin{lstlisting}
varref : T_ID index
    {
      symbol_t * sym = symbol_find (symtab, $1);
      assert (sym && "Ooops: Did not find variable!");
      /* You will add more code here */
      $$ = sym;
    }
\end{lstlisting}

Recall that the new rule should behave as follows:
\begin{itemize}
\item If the index actually appears, then some specific entry of the array 
must be read and stored in a temporary variable.
\item If no index is found, then the rule above should behave as a regular 
\texttt{OP\_LOAD} operation (e.g. \texttt{\_T10 = a}).
\end{itemize}

You can write your answer in words or in pseudocode, whatever is easier.

\begin{tcolorbox}[height=5in]

\end{tcolorbox}




\pagebreak


\paragraph{Problem 3: Writing to an array cell (10 points)}

As next step, we modify the assignment rule of grammar.y from the below form:

\vspace{2em}

\begin{lstlisting}

assignment : varref T_ASSIGN a_expr 
      {
        itab_instruction_add (itab, OP_STORE, $1->addr, $1->datatype, $3->addr);
        $$ = $1;
      }

\end{lstlisting}

\vspace{1em}

to its new form:

\vspace{1em}

\begin{lstlisting}
assignment : varref '[' a_expr ']' T_ASSIGN a_expr 
  {
    /* You will add more code here */ 
  }
\end{lstlisting}

\vspace{2em}

{\bf Problem 3A (3 points)}
\\
Briefly argue why we had to change the assignment rule in our grammar.
(HINT: the reason has to do with certain modifications to the non-terminal \texttt{varref}).

\begin{tcolorbox}[height=4in]

\end{tcolorbox}


\pagebreak
{\bf Problem 3B (4 points)}
\\
Now, briefly describe how would you implement a new 
{\bf \texttt{OP\_STORE\_ARRAY\_CELL}} intermediate code operation 
in the \texttt{run ()} function of icode.cc.
This operation is somewhat similar to \texttt{OP\_STORE} (See lines 144-150 in icode.cc), 
but it must store 
a single value in a particular array cell.
You can inspire yourself from your own answer of Problem 2B. Your
answer to this question must also be consistent with your answer in Problem 2A.

\begin{tcolorbox}[height=6in]

\end{tcolorbox}

\pagebreak

{\bf Problem 3C (3 points)}
\\
Finally, complete the semantic actions of the new assignment rule  below:

\begin{lstlisting}
assignment : varref '[' a_expr ']' T_ASSIGN a_expr 
  {
    /* You will add more code here */ 
  }
\end{lstlisting}

You must use the new intermediate code operation
{\bf \texttt{OP\_STORE\_ARRAY\_CELL}} in the above semantic action. You don't need to worry
about how are now regular assignments performed (e.g. \texttt{a := 10}).
You can write your answer in words or in pseudocode, whatever is easier.

\begin{tcolorbox}[height=6in]

\end{tcolorbox}



\pagebreak

\paragraph{Extra Credit Problem: Array Bounds Check (5 points)}

\paragraph{}
You are asked to add a safety feature in your compiler. The new feature
consists on checking that no out-of-bounds access is attempted on an array, neither when reading nor writing.
Notice that this \texttt{check} is performed at run-time, i.e. when the application
is running.

\paragraph{}
You have total freedom in deciding how to implement this feature. Just bear in mind 
that it must be a mix of collecting information at compile-time, and using that
information to perform the check at run-time.

\paragraph{}
Please try to be as concise as possible. Rambling will not get you too many extra points.

\begin{tcolorbox}[height=6.5in]

\end{tcolorbox}


\pagebreak


\end{document}
